\chapter{Introducción}
\label{chap:introducion}

\lettrine{E}{n} esta sección se describirá brevemente la temática de las prácticas. Además, se incluirán datos básicos y listas de objetivos generales y completos.

\section{Datos de la práctica}
\label{sec:datospractica}
\begin{itemize}
	\item {\bf Empresa:} Gradiant
	\item {\bf Duración:} 15/09/2025-17/10/2025
	\item {\bf Lugar de realización:}  Carretera do Vilar, 56-58, 36214, Vigo
	\item {\bf Tipo:} híbrido
	\item {\bf Remuneración económica:} 568 euros brutos/mes por 25 horas semanales 
	\item {\bf Departamento/equipo:} \textit{Identity \& Forensics}
	\item {\bf Tamaño departamento/equipo:} \~40	
	\item {\bf Cargo responsable de práctica:} Ingeniero-Investigador Senior 
\end{itemize}

\section{Objetivos}
\label{sec:obxectivos}

La temática de la práctica es el desarrollo de un sistema de detección de imágenes duplicadas en un conjunto de datos grande.

\subsection{Objetivos generales}
\begin{enumerate}
    \item Desarrollar un sistema de detección de imágenes duplicadas
    \item Integrar ese sistema en la plataforma existente
\end{enumerate}

\subsection{Objetivos concretos}
\begin{enumerate}
    \item Trabajar sobre un sistema de \textit{Image Retrieval} para indexar y consultar imágenes en una BDDD
    \item Experimentar con distintos modelos y técnicas de \textit{Computer Vision} 
    \item Evaluar los métodos propuestos
    \item Estructurar el código para prepararlo para la integración
\end{enumerate}