\chapter{Conclusiones}
\label{chap:conclusions}

\lettrine{E}{n} esta sección se aportarán unas conclusiones generales. Se describirán las relaciones con la titulación, y daré mi opinión general sobre del desarrollo de la práctica.

\section{Relaciones con la titulación}\label{sec:titulacion}

En la fase inicial \ref{subsec:actividade1} de familiarización con el código y lectura de \textit{papers}, traté con bastante material relacionado con la asignatura de Recuperación de Información y Minería Web. La principal diferencia con los sistemas que habíamos tratado en clase es que aquí los documentos eran imágenes, no texto, y podría haber estado bien estudiar algo de \textit{image retrieval} durante la asignatura. También aprendí sobre modelos basados en \textit{transformers} como los que habíamos visto en Aprendizaje Profundo.

En la fase de evaluación del sistema \ref{subsec:actividade2}, recordé lo estudiado en Fundamentos de Aprendizaje Automático y Modelos Avanzados de Aprendizaje Automático I y II al buscar datasets y métricas pertinentes para la tarea. Algunas de las soluciones alternativas también tocaban técnicas de Principios de Visión por Computador y Visión por Computador Avanzada, como SIFT, FFT y \textit{Haar Wavelets}. En la asignatura de VCA podría haber sido interesante revisar más en profundidad el \textit{SOTA} de \textit{Visual Transformers}, por ejemplo.

Para la fase final de integración \ref{subsec:actividade3} empleé conceptos vistos en Programación II, Ingeniería de Software y Herramientas de Desarrollo y Despliegue, como documentación adecuada y gestión correcta de git.

\section{Dificultades y retos}\label{sec:dificultades}

En cuanto a dificultades, he de admitir que familiarizarse con un campo nuevo como \textit{Image Retrieval} conllevó varios días de leer \textit{papers}, artículos y blogs. Por otro lado, tuve que aprender a usar Jira, una aplicación con infinidad de opciones, y adaptarme al \textit{tech stack} interno de la empresa para gestionar asuntos de RRHH, control horario y reuniones diarias. Ambos fueron difíciles durante las primeras semanas pero conseguí adaptarme correctamente.

\section{Valoración de la práctica}\label{sec:valoracions}

En general, puedo decir que estoy muy satisfecho con las prácticas. He descubierto y he aprendido a usar multitud de tecnologías que usaré en el futuro, y me he adentrado temporalmente en el mundo empresarial, lo cual considero beneficioso. 
