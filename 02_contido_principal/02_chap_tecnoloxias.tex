\chapter{Ferramentas e tecnoloxías}
\label{chap:tecnoloxias}

\lettrine{E}{n} esta sección se describirán algunas de las tecnologías que fueron utilizadas durante el desarrollo de la práctica. Serán acompañadas de ejemplos, explicaciones y citas bibliográficas.

\section{Ferramentas e tecnoloxías}\label{sec:tecnoloxias}

\subsection{Tecnoloxía 1: VSCode}\label{subsec:tecnoloxia1}

La elección del IDE (\textit{Integrated Development Environment}) para el desarrollo de las prácticas era libre, así que decidí utilizar VSCode por su versatilidad y mi experiencia con él. Existen IDEs con más \textit{features} para lenguajes concretos (Pycharm para Python, Visual Studio para C\#, etc.), pero VSCode puede funcionar para casi cualquier proyecto. Su \textit{marketplace} de extensiones permite transformar un simple editor de texto en la herramienta perfecta.

\subsection{Tecnoloxía 2: Github}\label{subsec:tecnoloxia2}

Existen varias plataformas de \textit{hosting} de repositorios de Git como Github, Gitlab o Bitbucket. Gradiant utiliza Github, que es probablemente la más popular de todas las opciones. Durante la duración de las prácticas fui contribuyendo a un repositorio para el proyecto, donde debí aplicar mis conocimientos prácticos sobre \textit{commits}, ramas y \textit{Pull Requests} para mantener un control de versiones adecuado. He de decir que la filosofía de Git en la empresa es más similar al \textit{Trunk-Based Development} \cite{Paul-Hammant_2020} que a \textit{Gitflow} \cite{Gitflow} que habíamos estudiado en clase, pero la adaptación no fue excesivamente complicada.

\subsection{Tecnoloxía 3: Jira}\label{subsec:tecnoloxia3}

Jira \cite{Jira_homepage} es una aplicación para gestión de proyectos donde un equipo de desarrolladores puede crear y administrar tareas enfocadas dentro del marco del desarrollo ágil de software \cite{TheAgileAlliance_2001}. Mi equipo utilizaba Jira para plasmar todas las tareas pendientes del proyecto y los estados en los que podían encontrarse: \textit{To-Do}, \textit{In Progress}, \textit{To Review} y \textit{Completed}.

% \begin{figure}[hp!] 
%   \centering
%   \includegraphics[width=0.65\textwidth]{04_imaxes/ejemplo_jira.png}
%   \caption{Ejemplo de un tablón de tareas de Jira}
%   \label{fig:jira}
% \end{figure}

No conocía esta tecnología, porque durante el grado habíamos usado otra herramienta \cite{Taiga} para gestionar los \textit{Sprints}, pero claramente resultó útil para orquestar el flujo de trabajo de un equipo de más de 10 personas.

\subsection{Tecnoloxía 4: Modelos CLIP}\label{subsec:tecnoloxia4}

Otra de las tecnologías nuevas que no había visto durante el grado es la familia de modelos multimodales CLIP. El modelo original fue lanzado por OpenAI en 2021 \cite{radford2021learningtransferablevisualmodels} y utiliza dos \textit{encoders} y entrenamiento de pares (imagen, texto) para intentar aprender la relación entre información visual y textual.

\begin{figure}[hp!] 
  \centering
  \includegraphics[width=0.75\textwidth]{04_imaxes/clip_model.png}
  \caption{Explicación del funcionamiento de CLIP}
  \label{fig:clip}
\end{figure}

El modelo Siglip2 fue lanzado por Google en 2025 \cite{tschannen2025siglip2multilingualvisionlanguage} y es una versión mejorada con capacidades plurilingües y mejor comprensión semántica. También experimenté con Siglip \cite{zhai2023sigmoidlosslanguageimage} y \textit{Perceptual Encoder} \cite{bolya2025perceptionencoderbestvisual}.

\subsection{Tecnoloxía 5: uv}\label{subsec:tecnoloxia5}

uv \cite{Astral_2025} es una herramienta ligera escrita en Rust para gestionar proyectos de Python. En el proyecto utilizamos esta tecnología para gestionar las dependencias y compilar el paquete.

\begin{lstlisting}[language=sh]
# crear .venv (si no existe) y sincronizar dependencias existentes
uv sync

# ejecutar script dentro de venv
uv run script.py

# compilar proyecto
uv build
\end{lstlisting}
